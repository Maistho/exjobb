\documentclass{sigchi}
\usepackage[utf8x]{inputenc}
\usepackage[T1]{fontenc}
\usepackage{lmodern}
\usepackage[swedish,english]{babel}

% Copyright statement
\toappear{Permission is granted to do whatever}


% Arabic page numbers for submission.
% Remove this line to eliminate page numbers for the camera ready copy
\pagenumbering{arabic}


% Load basic packages
\usepackage{balance}  % to better equalize the last page
\usepackage{graphics} % for EPS, load graphicx instead
\usepackage{times}    % comment if you want LaTeX's default font
\usepackage{url}      % llt: nicely formatted URLs

% llt: Define a global style for URLs, rather than the default one
\makeatletter
\def\url@leostyle{%
  \@ifundefined{selectfont}{\def\UrlFont{\sf}}{\def\UrlFont{\small\bf\ttfamily}}}
\makeatother
\urlstyle{leo}


% To make various LaTeX processors do the right thing with page size.
\def\pprw{8.5in}
\def\pprh{11in}
\special{papersize=\pprw,\pprh}
\setlength{\paperwidth}{\pprw}
\setlength{\paperheight}{\pprh}
\setlength{\pdfpagewidth}{\pprw}
\setlength{\pdfpageheight}{\pprh}

% Make sure hyperref comes last of your loaded packages,
% to give it a fighting chance of not being over-written,
% since its job is to redefine many LaTeX commands.
\usepackage[pdftex]{hyperref}
\hypersetup{
pdftitle={The effects of using Planning Poker on Code Quality},
pdfauthor={LaTeX},
pdfkeywords={Planning Poker, Code Quality, Agile},
bookmarksnumbered,
pdfstartview={FitH},
colorlinks,
citecolor=black,
filecolor=black,
linkcolor=black,
urlcolor=black,
breaklinks=true,
}

% create a shortcut to typeset table headings
\newcommand\tabhead[1]{\small\textbf{#1}}


% End of preamble. Here it comes the document.
\begin{document}

\title{The effects of using Planning Poker on Code Quality}

\numberofauthors{2}
\author{
  \alignauthor Gustav Bylund\\
  \affaddr{IDA, Linkoping University}\\
    \email{gustav.bylund@liu.se}\\
  \alignauthor Filip Lindman Marko\\
    \affaddr{IDA, Linkoping University}\\
    \email{filip.lindman.marko@liu.se}\\
}

\maketitle

\begin{abstract}
Planning poker is a technique for estimating the time required for tasks
within a programming project, which tries to eliminate certain biases of
the participants in order to achieve greater accuracy of the estimates\cite{grenning2002planning}.
This study aims to examine the effects of using Planning Poker within a project on the quality of code produced.
\end{abstract}

\keywords{
    Planning Poker; Code Quality; Agile
}

\section{Introduction}
Within agile development, a lot of factors come in to play when managing a project. One major factor is the use of user stories for planning the project.
Within agile development it has become more and more important to have accurate time estimates.
There has been some studies on the accuracy of estimates, where planning poker has been found to
possibly increase the accuracy of estimates\cite{molokken2007combining}.

Planning Poker isn't rated as one of the most important aspects of agile development\cite{Williams:2012:ATT:2133806.2133823}, and we ask ourselves why? Could developers not see the, perhaps hidden, beneficial aspects of Planning Poker, or is it simply so that developers dislike using Planning Poker?
We will therefore try to explore the possibly hidden benefits of using Planning Poker.


\section{Purpose}

This study will examine Planning Poker as a method to estimate time required for tasks within a programming project. It is to be investigated especially how the test participants experience the method, but also how the estimates compare to other methods.

\section{Research questions}
\begin{enumerate}
	\item How is code quality affected by using Planning Poker within a project?
	\item How do participants experience code quality changes from using Planning Poker?

	\item How accurate is the estimates of Planning Poker compared to <insert other method here>

\end{enumerate}

\section{Limitations}
The study will only involve two different methods of time estimation.

The study only involves one group of programmers.

\section{Background and related works}
In order to explore the relationship between the use of planning poker and code quality, we first need to define both concepts.

\subsection{Planning Poker}
Planning Poker was first introduced as a time estimation technique by James Grenning in [year] \cite{grenning2002planning}.
Grenning suggests that using less precise estimates and actively involve all programmers would
decrease time spent on estimation, and give more accurate estimates. The concept has since been
explored in several other projects and has been found to contribute to an increased accuracy of
estimates\cite{Mahni20122086,1667560}.

Planning poker has been suggested to have other benefits apart from accurate estimations, such as increased knowledge sharing and programmer enjoyment\cite{molokken2008using}. Use of Planning Poker has also been speculated to increase the quality of code produced\cite{molokken2007combining}.

\subsection{Code Quality}
Code Quality is a common name given to the measurement of software quality.
In order to perform software quality measurement, we will need some sort of explored standard.
The ISO/IEC 25010:2011 defines a quality in use model composed of eight characteristics\cite{iso250102011},
which we will use to analyse the quality of our code:
\begin{itemize}
\setlength{\itemsep}{0.25em}
\item functional suitability
\item reliability
\item performance efficiency
\item usability
\item security
\item compatibility
\item maintainability
\item portability
\end{itemize}

Agile teams often find the use of automated tools for code quality assurance important\cite{Williams:2012:ATT:2133806.2133823}, and we will try to evaluate how we can incorporate that into our testing\cite{ala2005survey}.

\section{Method}

We followed a team of X programmers during X weeks of development on a [briefly describe project]. During this time we acted as participating observers, taking part in group meetings and development.

The development period was divided in to week-long sprints. [describe sprints and tasks?] Task estimation and completion data was recorded for each sprint.
We have participated as active [observants] in a group of developers during a period of X weeks. Every week has been it's own development sprint, with task estimation and completion data recorded for each week.


Half of the sprints were initiated with a Planning Poker session. Remaining sprints utilised the team's pre-existing estimation technique [expert estimates?].

Each week was also concluded with a code review session, where we tried to measure code quality using the aspects described above.

We opted to alternate between the two methods every other week, in order to minimise the effects of [X bias]. This ensures that we would not be biased towards thinking that either method is better, only as a result of it being used earlier or later in the development process.
In order to measure actual time spent on tasks, each team member was asked to log his/her time spent using [time logging method]. This was then compared to the estimates, to produce the accuracy of estimates.

At the end of each development sprint each team member answered a survey, where they could rate their experience of the week and the work produced. They were also asked to estimate how accurate the estimates had been, and how enjoyable they found the planning method. The team members were also encouraged to comment with any additional thoughts on the process.
The results from the survey were anValysed to see if there was any trends among the team members.

\section{Conclusion}

Our conclusion is that...

\section{Acknowledgements}

Thanks to the team at Briteback for allowing us to use them as Guinea Pigs.

% Balancing columns in a ref list is a bit of a pain because you
% either use a hack like flushend or balance, or manually insert
% a column break.  http://www.tex.ac.uk/cgi-bin/texfaq2html?label=balance
% multicols doesn't work because we're already in two-column mode,
% and flushend isn't awesome, so I choose balance.  See this
% for more info: http://cs.brown.edu/system/software/latex/doc/balance.pdf
%
% Note that in a perfect world balance wants to be in the first
% column of the last page.
%
% If balance doesn't work for you, you can remove that and
% hard-code a column break into the bbl file right before you
% submit:
%
% http://stackoverflow.com/questions/2149854/how-to-manually-equalize-columns-
% in-an-ieee-paper-if-using-bibtex
%
% Or, just remove \balance and give up on balancing the last page.
%
\balance

% REFERENCES FORMAT
% References must be the same font size as other body text.

\bibliographystyle{acm-sigchi}
\bibliography{exjobb}
\end{document}
