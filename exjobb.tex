\documentclass{sigchi}

% Use this command to override the default ACM copyright statement (e.g. for preprints).
% Consult the conference website for the camera-ready copyright statement.


%% EXAMPLE BEGIN -- HOW TO OVERRIDE THE DEFAULT COPYRIGHT STRIP -- (July 22, 2013 - Paul Baumann)
% \toappear{Permission to make digital or hard copies of all or part of this work for personal or classroom use is 	granted without fee provided that copies are not made or distributed for profit or commercial advantage and that copies bear this notice and the full citation on the first page. Copyrights for components of this work owned by others than ACM must be honored. Abstracting with credit is permitted. To copy otherwise, or republish, to post on servers or to redistribute to lists, requires prior specific permission and/or a fee. Request permissions from permissions@acm.org. \\
% {\emph{CHI'14}}, April 26--May 1, 2014, Toronto, Canada. \\
% Copyright \copyright~2014 ACM ISBN/14/04...\$15.00. \\
% DOI string from ACM form confirmation}
%% EXAMPLE END -- HOW TO OVERRIDE THE DEFAULT COPYRIGHT STRIP -- (July 22, 2013 - Paul Baumann)


% Arabic page numbers for submission.
% Remove this line to eliminate page numbers for the camera ready copy
% \pagenumbering{arabic}


% Load basic packages
\usepackage{balance}  % to better equalize the last page
\usepackage{graphics} % for EPS, load graphicx instead
\usepackage{times}    % comment if you want LaTeX's default font
\usepackage{url}      % llt: nicely formatted URLs

% llt: Define a global style for URLs, rather that the default one
\makeatletter
\def\url@leostyle{%
  \@ifundefined{selectfont}{\def\UrlFont{\sf}}{\def\UrlFont{\small\bf\ttfamily}}}
\makeatother
\urlstyle{leo}


% To make various LaTeX processors do the right thing with page size.
\def\pprw{8.5in}
\def\pprh{11in}
\special{papersize=\pprw,\pprh}
\setlength{\paperwidth}{\pprw}
\setlength{\paperheight}{\pprh}
\setlength{\pdfpagewidth}{\pprw}
\setlength{\pdfpageheight}{\pprh}

% Make sure hyperref comes last of your loaded packages,
% to give it a fighting chance of not being over-written,
% since its job is to redefine many LaTeX commands.
\usepackage[pdftex]{hyperref}
\hypersetup{
pdftitle={SIGCHI Conference Proceedings Format},
pdfauthor={LaTeX},
pdfkeywords={SIGCHI, proceedings, archival format},
bookmarksnumbered,
pdfstartview={FitH},
colorlinks,
citecolor=black,
filecolor=black,
linkcolor=black,
urlcolor=black,
breaklinks=true,
}

% create a shortcut to typeset table headings
\newcommand\tabhead[1]{\small\textbf{#1}}


% End of preamble. Here it comes the document.
\begin{document}

\title{SIGCHI Conference Proceedings Format}

\numberofauthors{3}
\author{
  \alignauthor Gustav Bylund\\
    \affaddr{Affiliation}\\
    \affaddr{Address}\\
    \email{e-mail address}\\
    \affaddr{Optional phone number}
  \alignauthor Filip Lindman Marko\\
    \affaddr{Affiliation}\\
    \affaddr{Address}\\
    \email{e-mail address}\\
    \affaddr{Optional phone number}
}

\maketitle

\begin{abstract}
Planning poker is a technique for estimating the time required for tasks
within a programming project, which tries to eliminate certain biases of
the participants in order to achieve greater accuracy of the estimates.
This study aims to examine the participants perception of two different
methods for time estimation, and compare them to each other.
\end{abstract}

\keywords{
	Guides; instructions; author's kit; conference publications;
	keywords should be separated by a semi-colon. \newline
	\textcolor{red}{Optional section to be included in your final version,
  but strongly encouraged.}
}

\category{H.5.m.}{Information Interfaces and Presentation (e.g. HCI)}{Miscellaneous}

See: \url{http://www.acm.org/about/class/1998/}
for more information and the full list of ACM classifiers
and descriptors. \newline
\textcolor{red}{Optional section to be included in your final version,
but strongly encouraged. On the submission page only the classifiers’
letter-number combination will need to be entered.}

\section{Introduction}
Within agile development it has become more and more important to have accurate time estimates. There has been some studies on the accuracy of estimates, where planning poker has been found to possibly increase the accuracy of estimates.
But accuracy is not the only measurement by which an estimation technique can be valued. Other factors, such as group "togetherness" can affect "something". That's why we want to examine what other factors that affect participants experience of Planning Poker.


\section{Purpose}

This study  will examine Planning Poker as a method to estimate time required for tasks within a programming project. It is to be investigated especcialy how the test participants experience the method, but also how the estimates compare to other methods.

\section{Questions}
We want to examine:
\begin{itemize}
		\item Precision of estimates
		\item experienced work for participants
		\item participants appreciation of the method
		\item time spent estimating\cite{acrobat}
\end{itemize}
Of particular interest is examining if and how the accuracy of estimates affect participants appreciation of the method.

\section{Limitations}
The study will only involve two different methods of time estimation.
The study only involves one group of programmers.

\section{Background and related works}
There are…

Variations/types/... of planning poker
Example1, example2, these people said this about that.

Important things, observed improvements, stuff like that

References
[1] Grenning, James. "Planning poker or how to avoid analysis paralysis while release planning." Hawthorn Woods: Renaissance Software Consulting 3 (2002).
[2] Molokken-Ostvold, Kjetil, and Nils Christian Haugen. "Combining estimates with planning poker--an empirical study." In Software Engineering Conference, 2007. ASWEC 2007. 18th Australian, pp. 349-358. IEEE, 2007.
[3] Wellington, Carol A., Thomas Briggs, and C. Dudley Girard. "Examining team cohesion as an effect of software engineering methodology." In ACM SIGSOFT Software Engineering Notes, vol. 30, no. 4, pp. 1-5. ACM, 2005.:w

\section{Conclusion}

Our conclusion

\section{Acknowledgments}

Thanks to all of IP

% Balancing columns in a ref list is a bit of a pain because you
% either use a hack like flushend or balance, or manually insert
% a column break.  http://www.tex.ac.uk/cgi-bin/texfaq2html?label=balance
% multicols doesn't work because we're already in two-column mode,
% and flushend isn't awesome, so I choose balance.  See this
% for more info: http://cs.brown.edu/system/software/latex/doc/balance.pdf
%
% Note that in a perfect world balance wants to be in the first
% column of the last page.
%
% If balance doesn't work for you, you can remove that and
% hard-code a column break into the bbl file right before you
% submit:
%
% http://stackoverflow.com/questions/2149854/how-to-manually-equalize-columns-
% in-an-ieee-paper-if-using-bibtex
%
% Or, just remove \balance and give up on balancing the last page.
%
\balance

% REFERENCES FORMAT
% References must be the same font size as other body text.

\bibliographystyle{acm-sigchi}
\bibliography{exjobb}
\end{document}
